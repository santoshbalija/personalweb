%!TEX root = ..\arxiv_Dampinguncertanity_manuscript.tex
%-------------------------------
%*******************************
\section{INTRODUCTION} \label{sec:intro}
Damping estimation is a critical attribute to predict accurate dynamics of the system. There are several well-established methods to predict the damping in an oscillatory system. One of the well-known methods is logarithmic decrement~\cite{1890a}, which is initially developed for determining the viscosity of the fluids. Now, it is widely used in estimating the damping in structures~\cite{Fang1996}, cantilever beams~\cite{Liao2011}, plates~\cite{Saito1982}, wind turbines~\cite{Hansen2006}, and energy harvesters~\cite{Stanton2010}. The other methods of damping estimation are random decrement technique~\cite{Ibrahim1977}, weighted response-integral method~\cite{GAYLARD2001}, wavelet transforms~\cite{Ahmed2016}. Fang-LinHuang et al. (2007)~\cite{Huang2007} proposed a new method for identifying structural damping, which overcomes the limitations of the logarithmic decrement method. This method gives a better estimation of damping in noisy signals with a high damping ratio compared with the logarithmic method.

Recently, J A. Little \& B P. Mann~\cite{Little2019} derived an analytical expression for the optimal no of periods to be considered in the logarithmic decrement method to minimize the error in the predicted damping. Uncertainty analysis is performed to arrive at the optimal no of periods. In this paper, we develop guidelines to pick the optimal number of areas to be considered in Fang-LinHuang et al. (2007)~\cite{Huang2007} work. We perform uncertainty analysis to develop the guidelines. The new method for identifying structural damping~\cite{Huang2007} gives better results than the logarithmic decrement method, and we develop guidelines to pick the optimal no of periods to make this approach even more robust in estimating the damping ratio.

The subsequent sections of this work as follows. First, we give the background about minimizing the error in the logarithmic decrement method. Next, we perform the uncertainty analysis approach to the area ratio damping estimation method. We also perform a large number of numerical simulations to arrive at the optimal number of areas to be considered. We conclude with results of minimizing the error in the area ratio damping estimation method.