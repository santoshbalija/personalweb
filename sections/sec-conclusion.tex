%!TEX root = ..\arxiv_Dampinguncertanity_manuscript.tex
%-------------------------------
%*******************************
\section{CONCLUSION}
\label{sec:conclusion}
%*******************************
This study findings provide guidelines for choosing the optimal number of areas to be considered in the area ratio damping estimation method. The area ratio damping gives a better estimation of damping in high damping and noisy environments than the logarithmic decrement method. We developed guidelines to make the area ratio damping method for more robust damping estimation. We are not aware of any prior studies for finding the optimal number of areas in the area ratio damping estimation method.

The uncertainty analysis performed in this work gives more insights into choosing the optimal number of areas. Our guidelines suggest that for a very low damping ratio ($<0.01$), choosing more than two areas in the estimation increases the uncertainty.
In contrast, for moderate to high damping (between $0.05$ and $1$), we need to consider all the available areas in the estimation.

When we consider the uncertainties in zero-crossings, the analysis becomes significantly complicated. Therefore we have not included uncertainties in zero-crossings. We left the uncertainty analysis with considering zero-crossings  as future work.