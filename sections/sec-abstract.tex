%!TEX root = ..\arxiv_Dampinguncertanity_manuscript.tex
%-------------------------------
%*******************************
The logarithmic decrement (log-dec) is one of the most popular methods for viscous damping estimation in linear, single degree of freedom systems. 
It estimates the damping ratio by examining the decay in the amplitude between two peaks some number of cycles apart. 
The accuracy in the estimation is sensitive to the chosen number of cycles, where the latter can be optimized such that the uncertainty in the estimation is minimized. 
However, the log-dec method is not suitable for systems with high damping ratios (approximately $>0.3$).  
Another recent approach for damping estimation is based on considering a ratio of the amplitudes of the positive and negative areas in the free response of the oscillator. 
Although prior works on the areas method only tested lightly damped systems, we show here that---in contrast to log-dec---this approach can estimate the damping ratio over the whole range of underdamped linear oscillators. 
However, in contrast to log-dec, there are no available guidelines on how many areas to include in the damping estimation. 
In this work, we derive uncertainty analysis expressions for the areas method and we utilize them to obtain the optimal number of areas to use. 
Our results show that for a very low damping ratio ($<0.01$), choosing more than two areas in the estimation increases the uncertainty. 
In contrast, for moderate to high damping (between $0.05$ and $1$), we need to consider all the available areas in the estimation. 
One caveat in the range of high damping (between $0.3$ and $1$) is that while it is desirable to include all the available areas, uncertainty increases when considering up to 3 areas. 
Therefore, if only 4 areas are available in this range, then to reduce the uncertainty in the estimate only the first two areas must be considered. 
The results are verified using a large number of numerical simulations including different levels of noise. 